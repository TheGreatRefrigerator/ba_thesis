\section{Methodik}
Die Grundsätzlichen Überlegungen beruhen, wie in der Einleitung bereits erwähnt, auf den Notstandsvollmachten durch §35 Absatz 1 der StVO für Einsatzfahrzeuge des Rettungsdienstes, der Feuerwehr und der Polizei.
Demnach dürfen sich Einsatzfahrzeuge im Notfall über die Vorschriften der StVO hinwegsetzen.

Die Frage, die sich hierbei stellt, ist, inwieweit von diesen Rechten im Ernstfall Gebrauch gemacht werden kann.
Um diesen Sachverhalt zu untersuchen, wurde ein Fragenkatalog für die \gls{ffl} zusammengestellt.
Dieser wurde als Datei in der Cloud gespeichert und ist auch weiterhin verfügbar (Link im Anhang auf S.\pageref{sec:anhang}).
Die Fragen wurden unter Rücksicht der \gls{osm}-Datenstruktur und das vorhandene ORS Backend gewählt.
Die Antworten der \gls{ffl} sind hier sinngemäß wiedergegeben und die wörtlichen Antworten sind im Cloud-Dokument im Anhang zu entnehmen.
Auf Basis der erhobenen Informationen wurden geeignete \gls{osm}-Tags gesucht mit denen die empfohlenen Änderungen verwirklicht werden können.
Anschließend wurden neue Java-Funktionen geschrieben um diese Tags bei der Erstellung des Graphen und bei Anfragen zu berücksichtigen.
Außerdem wurde das Frontend an das neue Profil angepasst.

\subsection{Informationserhebung}

\paragraph*{1. Maße der Fahrzeuge:}
\label{frage1}
\par
Die Dimensionen des Fahrzeuges sind limitierende Faktoren für bestimmte Wegsegmente.
Manche Brücken halten nur ein bestimmtes Gewicht aus und ein Tunnel hat nur eine gewisse Höhe.
Hier stützt sich das \gls{ors} Backend auf Restriktionen durch die \gls{osm}-Tags: \texttt{maxlength, maxwidth, maxheight, maxweigth} und \texttt{maxaxleload} Von der \gls{ffl} wurde als wichtigstes Fahrzeug Löschfahrzeuge der Klasse LF8, LF8/6 und MLF mit folgenden Daten angegeben:
\begin{itemize}
\centering
\item[Länge:] 7 Meter
\item[Breite:] 2,5 Meter
\item[Höhe:] 3 Meter
\item[Gewicht:] 7,5 Tonnen
\end{itemize}

\paragraph*{2. Maximale Geschwindigkeiten auf unterschiedlichen Straßentypen:}
\label{frage2}
\par
Es kann nicht auf jeder Straße mit maximaler Geschwindigkeit des Fahrzeugs gefahren werden.
Deswegen werden für jede Kante des Graphen Geschwindigkeitsmaxima definiert.
Dabei wird auf einen Standardwert für den Straßentyp, die \textit{Values}(Werte) des \gls{osm} \textit{Keys} (Schlüssel) \texttt{highway}, zurückgegriffen.
Tabelle \ref{tab:speedinfo} zeigt die bisherigen Standardgeschwindigkeiten für das Schwerfahrzeug-Profil des \gls{ors} zusammen mit den vorgeschlagenen Werten der \gls{ffl} in Klammer.
Es wurde darauf hingewiesen, dass die 80km/h auf Autobahnen nur gelten wenn kein Stau besteht.
Im Fall eines Staus wäre laut \gls{ffl} auch die Anfahrt auf der Gegenfahrbahn bzw. entgegen der Verkehrsrichtung auf der gleichen Bahn interessant.

Genaue Informationen zu einzelnen Tags können der \gls{osm} Wikipedia (\cite{osmwiki}) entnommen werden.

\begin{table}[htb]
\centering
\caption{Standardgeschwindigkeiten für unterschiedliche \texttt{highway} Values}
\label{tab:speedinfo}
\begin{tabular}{|l|r|}
\hline
\multicolumn{1}{|l|}{\texttt{highway} Value} & \multicolumn{1}{c|}{\textit{km/h}} \\
\hline
\multicolumn{2}{|l|}{\textit{Autobahnen}} \\
\hline
motorway (Autobahn) & 80 \\
motorway\_link (Autobahn-Zubringer) & 50 \\
motorroad (Kraftfahrtstraße) & 80 \\
trunk (Schnellstraße) & 80 \\
trunk\_link (Zubringer zu Schnellstraße) & 50 \\
\hline
\multicolumn{1}{|l}{\textit{Siedlungen}} & \multicolumn{1}{l|}{ } \\
\hline
primary (Bundesstraßen) & 60(80) \\
primary\_link  & 50 \\
secondary (Landes-/Staatsstraßen) & 60(80) \\
secondary\_link  & 50 \\
tertiary (Kreisstraßen) & 60(80) \\
tertiary\_link  & 50 \\
unclassified (Nebenstraße (oft ohne Mittellinie))  & 60 \\
residential (Straße an und in Wohngebieten) & 50 \\
living\_street (Spielstraße)  & 10 \\
service (Erschließungsstraße) & 20 \\
road (unbekannte Straße) & 20 \\
track (Wald-/Feldweg) & 15 \\
\hline
\end{tabular}
\end{table}

Auf Wirtschafts-, Wald- und Feldwegen mit dem Key-Value-Pair \texttt{highway=track} kann zusätzlich der Zustand des Weges mit dem Tag \texttt{tracktype} beschrieben werden.
Die bisherigen und die von der \gls{ffl} vorgeschlagenen Geschwindigkeit für die jeweiligen Values sind in Tabelle \ref{tab:speedinfotrack} zu sehen.

\begin{table}[h]
\centering
\caption{Standardgeschwindigkeiten für unterschiedlichen \texttt{tracktype} Values}
\label{tab:speedinfotrack}
\begin{tabular}{|l|r|}
\hline
\texttt{tracktype} Value & \textit{km/h} \\
\hline
grade1 (Wasserfester Belag) & 20(25)   \\
grade2 (Wassergebundene Decke) & 15  \\
grade3 (Befestigter oder ausgebesserter Weg) & 10(15)  \\
grade4 (Unbefestigter Weg) & 5(10)   \\
grade5 (Unbefestigter Weg) & 5   \\
\hline
\end{tabular}
\end{table}

\paragraph*{3. Dürfen vorgegebene Geschwindigkeiten (Bsp. 30er Zone/ Tempolimit 70 etc) überschritten werden?}
\label{frage3}
\par
Manchmal sind besondere Geschwindigkeitsrestriktionen, wie zum Beispiel Tempo-30-Zonen vorhanden.
In Tempo-30-Zonen darf 50 km/h und in Spielstraßen darf 20 km/h gefahren werden.
Dabei ist immer noch auf die Unversehrtheit der anderen Verkehrsteilnehmer (vor allem Kinder) zu achten.

\paragraph*{4. Dürfen Gewicht und/oder Achslast für Straßen mit Vorgabe überschritten werden?}
\label{frage4}
\par
Es dürfen Straßen benutzt werden die zum Beispiel wegen einer folgenden Brücke auf 7,5 Tonnen beschränkt sind.
Um die Brücke selbst zu nutzen ist allerdings lokales Wissen über die tatsächliche Tragfähigkeit notwendig.
Allgemein müssen bei Zu- oder Durchfahrten für die Feuerwehr die Aufstellflächen und die Bewegungsflächen so befestigt sein, dass sie von Feuerwehrfahrzeugen mit einer Achslast bis zu 10 t und einem zulässigen Gesamtgewicht bis zu 16 t befahren werden können.

\paragraph*{5. Dürfen Einbahnstraßen in Entgegengesetzter Richtung durchfahren werden?}
\label{frage5}
\par
Einbahnstraßen dürfen in entgegengesetzter Richtung durchfahren werden.

\paragraph*{6. Ist 5. im Einsatz überhaupt sinnvoll? (falls z.B. die Straße durch Fahrzeug(e) blockiert ist?}
\label{frage6}
\par
Ob eine Einbahnstraße benutzt wird muss der Fahrer während dem Einsatz entscheiden.
Es hängt dabei von der verfügbaren Breite und der Länge der Straße ab.
Auch die Möglichkeit des Ausweichens bei Gegenverkehr spielt eine Rolle.
Anhand dieser Faktoren wird entschieden ob Zeit gewonnen wird und die Einbahnstraße benutzt wird oder nicht.

\paragraph*{7. Zu welchen weiteren Routen bzw. Verkehrsnetz bezogenen Besonderheiten kommt es im Einsatz?}
\label{frage7}
\par
Es dürfen Bus-, Taxi- und Tram-Spuren verwendet werden.

\paragraph*{8. Ist es wichtig auf der richtigen Straßenseite anzukommen ?}
\par
Nein.

\paragraph*{9. Wäre eine Suche nach Hydranten, Löschwassertanks etc. am Zielort bzw. im Einzugsgebiet sinnvoll?}
\label{frage9}
\par
Vor Ort gibt es nicht immer Löschwasser, weshalb eine Suche nach Hydranten, Seen, Tanks, Flüssen und ähnlichen Quellen auf jeden Fall sinnvoll ist.

\paragraph*{10. Dürfen folgende Wegtypen befahren werden? Wenn ja mit welcher Geschwindigkeit?}
\label{frage10}
\par
Es dürfen die in Tabelle \ref{tab:speedinfospecial} gelisteten, für den normalen Straßenverkehr nicht zugänglichen Wegtypen benutzt werden.

\begin{table}
\caption{Standardgeschwindigkeiten für spezielle Wegtypen}
\label{tab:speedinfospecial}
\centering
\begin{tabular}{|l|c|}
\hline
\multicolumn{1}{|l|}{Key-Value-Paar} & \multicolumn{1}{c|}{Geschwindigkeit} \\
\hline
aeroway=runway (Start/Landebahn) & 80 \\
aeroway=taxilane (Rollweg) & 80 \\
highway=raceway (Rennstrecke) & 80 \\
highway=cycleway (Radweg) & 10 \\
tracktype=pedestrian(Fußgängerzone) & 10\\
tracktype=footway(Fußweg) & 5\\
\hline
\end{tabular}
\end{table}



\subsection{Generierung des Routing-Profils}
\label{backendGraphBuild}
Aufbauend auf den gewonnen Informationen aus dem Fragenkatalog können entsprechende Java-Klassen für das Löschfahrzeug-Profil erstellt werden.
Dazu werden die bereits vorhandenen Klassen eines \gls{ors}-Profils verwendet.
Da die Dimensionen des Fahrzeuges mitberücksichtigt werden sollen, wird das Heavy-Vehicle-Profil als Grundlage benutzt.
Die Prozessierung wird von vier Klassen gehandhabt, die an den passenden Stellen im Backends eingebunden werden müssen.

\begin{listing}[htb]
\centering
\RecustomVerbatimEnvironment{Verbatim}{BVerbatim}{}
\inputminted[fontsize=\footnotesize, breaklines, breakanywhere, firstline=26, lastline=53]{java}{../data/JavaFiles/EmergencyVehicleAttributesGraphStorage.java}
\caption{Speicherobjekt für die Fahrzeugeigenschaften}
\label{lst:storage}
\end{listing}

Davon sind drei für die Fahrzeugdimensionen zuständig:
\newline
Die Klasse \texttt{EmergencyVehicleAttributesGraphStorage} initiiert das Speicherobjekt und ist für die Fehlerbehandlung zuständig (Sourcecode~\ref{lst:storage}).
\newline
Die Klasse \texttt{EmergencyVehicleGraphStorageBuilder} baut das Speicherobjekt zusammen.
Hier werden die Beschränkungen für die Höhe, Breite, Länge, Achslast und das Gewicht aus den \gls{osm}-Daten gespeichert (Sourcecode~\ref{lst:builder}).
Das Profil für Schwerfahrzeuge berücksichtigt hier normalerweise unterschiedliche Restriktionen für verschiedene Fahrzeugtypen.
Manche Straßen sind beispielsweise nur für Forst- andere nur für Lieferfahrzeuge freigegeben.
Andere Straßen dürfen normalerweise aufgrund des \gls{osm}-Tags \texttt{access=private} oder \texttt{access=restricted} gar nicht verwendet werden.
Außer den materiellen Beschränkungen durch die Fahrzeugdimensionen gelten für Einsatzfahrzeuge diese Beschränkungen nicht.

\begin{listing}[htb]
\centering
\RecustomVerbatimEnvironment{Verbatim}{BVerbatim}{}
\inputminted[gobble=6, fontsize=\footnotesize, breaklines=true, breakbytoken=|, firstline=161, lastline=180]{java}{../data/JavaFiles/EmergencyVehicleGraphStorageBuilder.java}
\caption{Entnahme der Dimensionsbeschränkungen}
\label{lst:builder}
\end{listing}

Die Klasse \texttt{EmergencyVehicleEdgeFilter} vergleicht bei der Ausführung der Routing-Anfrage für jede Kante die als Parameter angegebenen Fahrzeug-Dimensionen mit den Restriktionen aus dem im \texttt{EmergencyVehicleGraphStorageBuilder} produzierten Objekt.
Liegt der übergebene Wert unterhalb der Beschränkung kann potenziell über diese Kante geroutet werden.
Überschreitet der Wert die Beschränkung, wird die Kante vom Routing ausgeschlossen (Sourcecode~\ref{lst:filter}).

\begin{listing}[htb]
\centering
\RecustomVerbatimEnvironment{Verbatim}{BVerbatim}{}
\inputminted[gobble=5, fontsize=\footnotesize, breaklines=true, breakbytoken=|, firstline=102, lastline=107]{java}{../data/JavaFiles/EmergencyVehicleEdgeFilter.java}
\caption{Vergleich der übergebenen Dimensionen mit den Restriktionen des Graphen}
\label{lst:filter}
\end{listing}

Der Hauptteil der Änderungen wird in der \texttt{EmergencyFlagEncoder} Klasse vorgenommen.
Hier werden beispielsweise die Maximalgeschwindigkeiten für die unterschiedlichen Wegtypen (Frage 2.) implementiert (Abb. ~\ref{lst:flag}).
Die Geschwindigkeiten sind bewusst höher als die zurückgegebenen Werte gewählt, mit der Absicht das Löschfahrzeug-Profil auch für andere Einsatzfahrzeuge, wie zum Beispiel PKW der Polizei nutzbar zu machen.
Das Geschwindigkeitslimit von 80km/h hängt mit der ausgewählten Fahrzeugklasse zusammen und nicht mit der Straßenart.
Fahrradwege und Fußgängerzonen werden durch die Zuweisung einer Geschwindigkeit ebenfalls für das Profil nutzbar gemacht.

\begin{listing}[htb]
\centering
\RecustomVerbatimEnvironment{Verbatim}{BVerbatim}{}
\inputminted[gobble=2, fontsize=\footnotesize, breaklines, breakbytoken=|, firstline=169, lastline=199]{java}{../data/JavaFiles/EmergencyFlagEncoder.java}
\caption{Definition von Maximalgeschwindigkeiten für unterschiedliche Wegtypen}
\label{lst:flag}
\end{listing}

Die Beschränkungen 30er-Zonen und Spielstraßen sollen mit dem Schnipsel aus Code-Beispiel \ref{lst:zone} übergangen werden.
Wichtig dabei ist, dass der \gls{osm}-Tag \texttt{highway=living\_street} nicht automatisch eine Spielstraße ist.
In Spielstraßen darf in Deutschland nur Schrittgeschwindigkeit gefahren werden.
Diese Eigenschaft muss beim \textit{Tagging}\footnote{taggen: das Zuweisen eines Key-Value-Pairs zu einem \gls{osm} Objekt} berücksichtigt werden.
Häufig wird solchen Straßen der Tag \texttt{maxspeed=7} zugewiesen.


Um Bus- und Taxispuren für das Profil nutzbar zu machen, wurde das Code Segment \ref{lst:bus} implementiert.
Für mit Fahrzeugen befahrbare Tramspuren gibt es in \gls{osm} bisher bis dato keinen geeigneten Tag.
Die Segmente der Tramspur, die befahren werden können, liegen in der Regel auf einer vorhandenen Straße.
\begin{listing}[htb]
\centering
\RecustomVerbatimEnvironment{Verbatim}{BVerbatim}{}
\inputminted[gobble=2, fontsize=\footnotesize, breaklines=true, breakbytoken=|, firstline=299, lastline=303]{java}{../data/JavaFiles/EmergencyFlagEncoder.java}
\caption{Limit für 30er-Zonen und Spielstraßen}
\label{lst:zone}
\end{listing}

Um Einbahnstraßen in jeder Richtung nutzen zu können, wurden Teile des Codes entfernt, die einer Kante eine vorwärts- oder rückwärts-Richtung geben\footnote{Jeweils aus Sicht der Startecke einer Kante gesehen}.

Der \texttt{EmergencyFlagEncoder} beinhaltet desweiteren die Funktion \texttt{collect}, welche dem Wegsegment anhand seines Wegtyps eine Priorität zugeschreibt.
Auf diese Weise können Straßen auf denen hohe Geschwindigkeiten gefahren werden können oder die extra für Liefer- oder Forstfahrzeuge ausgeschrieben sind, bevorzugt benutzt werden und Straßen durch Wohngebiete oder Spielstraßen nur dann verwendet werden, wenn das Ziel anders nicht zu erreichen ist.

\begin{listing}[htb]
\centering
\RecustomVerbatimEnvironment{Verbatim}{BVerbatim}{}
\inputminted[gobble=2, fontsize=\footnotesize, breaklines=true, breakbytoken=|, firstline=403, lastline=417]{java}{../data/JavaFiles/EmergencyFlagEncoder.java}
\caption{Nutzung von speziellen Wegtypen}
\label{lst:bus}
\end{listing}

Durch den Code-Ausschnitt in Beispiel \ref{lst:weight} soll die höchste Priorität jene Straßen mit dem Tag \texttt{service=emergency\_access} erhalten.
Das sind zum Beispiel Rettungszufahrten zu Gebäuden und gesonderte Autobahnauffahrten, die vom normalen Verkehr nicht benutzt werden dürfen.
Da Busspuren auch vom motorisierten Individualverkehr freigehalten werden müssen, werden diesen an dieser Stelle ebenfalls eine höhere Priorität zugewiesen.

Beim Ausführen des Codes werden die auf diese Weise generierten Attribute (Flags) beim Bau des Graphen zusammen mit den geographischen Koordinaten, den Wegtypen, der Richtung, der Oberfläche und noch vielen weiteren Attributen auf den Kanten des Graphen gespeichert.
Der Graph wird, wie in Kapitel ~\ref{sec:osmgraph} bereits beschrieben, aufgebaut.

\begin{listing}[htb]
\centering
\RecustomVerbatimEnvironment{Verbatim}{BVerbatim}{}
\inputminted[gobble=1, fontsize=\footnotesize, breaklines=true, breakbytoken=|, firstline=562, lastline=567]{java}{../data/JavaFiles/EmergencyFlagEncoder.java}
\caption{Nutzung von Notfalleinfahrten}
\label{lst:weight}
\end{listing}

Um das Profil auch für die Kooperationspartner in Lützelburg nutzbar zu machen, wurde eine Instanz des \gls{ors} Backends mit diesen Änderungen auf einen Server der Universität Heidelberg installiert.
Genaue Anweisungen zum Aufsetzen so eines Servers werden in \cite{neisdoktor} ausführlich beschrieben.

Damit ist Prozessierung des Routing-Profils für Einsatzfahrzeuge fertiggestellt und es können Anfragen an die Schnittstelle des \gls{ors} gesendet werden.


\subsection{Anpassungen Frontend}

Das \gls{ors} Frontend ist ein \gls{gui}, welches die interaktive Erstellung einer Abfrage an das Backend ermöglicht.
Gleichzeitig kann das Ergebnis auf der gleichen Oberfläche dargestellt werden.
Das Frontend für das Löschfahrzeug-Profil kann unter dem Link \href{emergency.openrouteservice.org} erreicht werden.

\bigskip

Allgemein kann über einen Rechtsklick auf die Karte ein Start-, Weg- oder Endpunkt platziert werden.
Sobald zwei Punkte vorhanden sind, wird ein Routing-Anfrage für das ausgewählte Profil an das Backend gesendet und die erhaltene Antwort als schnellster Weg zwischen den Punkten dargestellt.
Genauere Informationen zu der Route, wie die benötigte Zeit, die Distanz oder die einzelnen Wegstücke können der Seitenleiste entnommen werden.
Um eine Erreichbarkeitsanalyse durchzuführen kann am linken Rand der Website von Routing auf Isochronen umgestellt werden.
Hier kann mit einem Rechtsklick auf die Karte ein Zentrum gesetzt werden.
Nach einem Klick auf den \texttt{Isochronen-Generieren-Button} wird die Antwort für den aus den Einstellungen generierten Anfrage angezeigt.

\bigskip

Da der Fokus der Arbeit auf der Erstellung des Profils und nicht der Darstellung der Ergebnisse liegt, werden die Änderungen am Frontend hier nur grob dargestellt.
Alle Änderungen können im Detail nachvollzogen werden, da der Quellcode online verfügbar ist (Siehe Anhang S.~\pageref{sec:anhang}). Das Auto- und Heavy-Vehicle-Profil wurde für direkte Vergleiche zu den Profil für Einsatzfahrzeuge beibehalten.
Gleichzeitig wurde das neue Profil für Einsatzfahrzeuge eingefügt für welches zwei Fahrzeugklassen vorhanden sind.

\bigskip

1. Profil für Löschfahrzeuge: Für dieses Profil werden automatisch die Höhe, Länge, Breite, Achslast, das Gewicht sowie die Höchstgeschwindigkeit eingestellt.
Damit müssen diese Daten bei einem neuen Aufruf der Seite nicht jedes mal neu eingegeben werden.
2. Profil allgemein für Einsatzfahrzeuge: Hier werden Fahrzeugdimensionen nicht automatisch eingestellt.
Die Höchstgeschwindigkeit für dieses Profil ist standardmäßig auf 130km/h gesetzt.
Diese kann in den weiteren Einstellungen deaktiviert werden.

\bigskip

Aufgrund der Berechnungszeit für den Graphen wurde der Datenbereich auf den politischen Raum von Schwaben begrenzt.
Abfragen können nur in diesem Bereich getätigt werden, weshalb dieser als durchsichtiges Polygon über der Karte visualisiert wird.
Der aktive Datenbereich kann über eine Checkbox in der rechten unteren Ecke angeschaltet werden.
