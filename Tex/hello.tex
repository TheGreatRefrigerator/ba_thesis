\documentclass[10pt,a4paper]{article}
\usepackage[utf8]{inputenc}
\usepackage[german]{babel}
\usepackage[T1]{fontenc}
\usepackage{amsmath}
\usepackage{amsfonts}
\usepackage{amssymb}
\usepackage{color}
\usepackage{hyperref}
\author{Amandus Stefan Butzer}
\title{Erstellung eines Routing-Profils auf Basis von OSM / Öffentlichen Daten für Feuerwehrfahrzeuge}
\date{\today}

\begin{document}

\maketitle

(\section{Danksagung})

\section{Kurzfassung}

\section{abstract}

\section{Einleitung}

Zum Zeitpunkt der Erstellung dieser Arbeit ist der Verfasser als Hilfswissenschaftler in der Geoinformatik Abteilung des Geographischen Instituts der Ruprecht-Karls-Universität Heidelberg als Mitarbeiter des \href{http://www.openrouteservice.org}{openrouteservice}(ORS) tätig. Der ORS bietet neben einem Geocoding, Routing und Locations Service auch einen Isochrones Service an. Immer wieder wurden Anfragen bezüglich Erreichbarkeitsanalysen aus dem Rettungs- und Brandschutzwesen erhalten. Für Polizei, Rettungsdienst und Feuerwehr geht es vor allem um das Einhalten amtlich vorgegebener Hilfsfristen. Im Gegensatz zum Rettungsdienst, welcher in den Zuständigkeitsbereich der Bundesländer fällt, ist der Brandschutz eine kommunale Aufgabe und unterliegt nur in manchen Bundesländern bestimmten Standards \cite{brandschutz bedarfsplan}.

Da mit dem Isochrones-Service des ORS Erreichbarkeitsanalysen durchgeführt werden, kann dieser für die Erstellung eines Brandschutzbedarfsplans genutzt werden. Bemängelt wurde allerdings, dass die Einsatzfahrzeuge in der Regel weit über das berechnete Gebiet für eine gegebene Zeit hinauskommen.

In dieser Arbeit wurde daher ein Emergency-Routing-Profil in Kooperation mit der \href{https://www.feuerwehr-luetzelburg.de}{Freiwilligen Feuerwehr Lützelburg} entwickelt. Die Implementierung ist auf eine Fahrzeugklasse der Feuerwehr begrenzt. Das Profil wurde derart gestaltet, dass Erweiterungen für diverse Einsatzfahrzeuge sehr einfach möglich sind, die im Rahmen dieser Batchelor Arbeit aber nicht erarbeitet wurden. 

Das ..... basiert auf dem ORS Backend, welches um die für das zusätzliche Profil benötigten Java Klassen erweitert wurde. Zur Darstellung wurde das ORS Frontend angepasst. 
weitere Infos über technic details? 

\subsection{Motivation}

STVO
Anwendbar für fast alle Einsatzfahrzeuge 

\section{Theoretische Grundlagen}

\subsection{Graphen Erstellung}

Graphen allgemein,

\subsection{Routing}

kürzester weg 
algorithmus Dijkstra + contraction hierarchies

\subsection{Isochronen Berechnung}

marching squares grid based , delauny erklären shape based(triangles) + unsere implementierung shape based (points)

grid based:
 grid is formed around the center point. depending on the mesh size it is more accurate or more coarse.
 The vertices on the grid get values depending on the snapped distance to the roadgraph.
 Points are marked on the edges between two vertices, where the value complies with the requested value.
 
Advantages of this algorithm : can be really fast and cover large distances and long timespans. can contain holes
Disadvantage: not very accurate in coarse mode. with fine grained grid accurate but high ressource consumption.

TIN:
TODO

Shape based:
- Uses Dijkstra algorithm to mark all reachable Edges
- Extract points from edge geometry(real world geom, becaus in graph we only have simple straight lines between nodes, geometry is saved in the graph edge as parameter)
- invisible bubbles (buffer) will be drawn around points so close points can be skipped
- build a point cloud with remaining points
- use Alpha shapes algorithm to draw isochrones around the point line

 Distance limitation 100km. resulting in different time limitation for different profiles (1hour car , 5hours bike, 20 hours foot)

Advantage: accurate, fast computation time
Disadvantage: Alpha shape library -> no holes.




Max erklären lassen
Einfache erklärung !

\section{Generierung des Routing-Profils}

\subsection{Informations Erhebung}
Fragebogen für Feuerwehr Lützelburg\footnote{Lützelburg ist eine stadt in Bayern}

\subsection{Limitierende Faktoren}

\subsection{Erweiternde Faktoren}

\section{Ergebnisse}
\paragraph{
Vergleiche zwischen Firetruck - Emergency Vehicle - Car - Heavy Vehicle
}
\paragraph{
\color{red}
Hier würde ich ein paar räumliche Beispiele aussuchen und exemplarisch zeigen (Routing und Isochronen), welche Änderungen das Profil mit sich bringt, einerseits innerstädtisch, andererseits auch außerhalb der Stadt. Denn Änderungen als solches ist bisschen schwierig zu definieren. Gerne die Jungs aus Lützelburg fragen, welches Gebiet mit den bereits vorhandenen Profilen wirklich schlechte Ergebnisse bringt und jetzt mit Emergency weitaus realitischere!
}

\section{Fazit}

tolles teil

\section{Future Work/Ausblick}
\begin{itemize}
\item Suche nach Löschwasser quellen um den Zielpunkt (osm tag emergency=fire_hydrant)
\item Beschleunigung
\item rush hour / Tag & Nacht Unterscheidung (nachts weniger los auf Straßen/ Fußgängerzonen ...)
\end{itemize}


\section{Literatur}

http://www.geog.uni-heidelberg.de/md/chemgeo/geog/gis/corp07-aas-pn-az-final.pdf
http://neis-one.org/wp-content/uploads/2010/05/Diplomarbeit_Neis.pdf
Edsger W. Dijkstra. A note on two problems in connexion with graphs. Numerische Mathematik,
1:269–271, 1959. (http://www-m3.ma.tum.de/foswiki/pub/MN0506/WebHome/dijkstra.pdf)
http://www.agbf.de/pdf/Fortschreibung%20der%20Empfehlung%20der%20Qualitaetskriterien%20fuer%20die%20Bedarfsplanung%20in%20Staedten%20Layout%20neu%202016.pdf


\end{document}
