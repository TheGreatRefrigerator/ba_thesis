\section{Fazit}

Das erstellte Profil für Löschfahrzeuge wird als Prototyp betrachtet und kann bereits die grundlegenden Anforderungen des \gls{ffl} erfüllen.
Es werden bei der Suche nach dem kürzesten Weg in einem Straßennetz die, für andere Fahrzeuge geltenden, Geschwindigkeitslimits auf geeignetere Werte angehoben.

Es ist für das Profil möglich, Einbahnstraßen in beide Richtungen zu befahren.
Dadurch kann die Distanz zum Ziel signifikant reduziert werden.
Das führt aber auch dazu, dass häufig über längere Strecken auf der Gegenfahrbahn gefahren wird und Abbiegespuren des Gegenverkehrs, wegen wenigen gewonnenen Metern, verwendet werden.
Infolgedessen sollte ein Penalty-Faktor eingebaut werden, der bei einer Nutzung entgegen der Fahrtrichtung eine Verzögerung mit einberechnet.

Darüber hinaus werden bisher nicht befahrbare Straßentypen wie Fußgängerzonen und Radwege für das Profil verfügbar gemacht.
Eine Routenführung bis zum Zielpunkt wird so gegenüber den normalen Fahrzeug-Profilen in den meisten Fällen erreicht.

Derzeit wird die Schwerfälligkeit des Löschfahrzeuges durch positive Faktorisierung\footnote{Die Gewichtung wird mit Werten größer 1.0 multipliziert} der Gewichtung für Start, Ankunft und Abbiegevorgängen, bei der Suche nach dem schnellsten Weg, berücksichtigt.
Anhaltspunkte hierfür wurden mit Hilfe der \gls{ffl} bestimmt.
Für die Teststrecken wurde beim Routing von Start bis Endpunkt jeweils die richtige Route zurückgegeben.
Ob diese Faktorisierung allgemein gültig ist, muss durch weitere Analysen überprüft werden.
Für die Berechnung der Zeit werden, zur Berücksichtigung der Schwerfälligkeit, Penalties für Start, Ankunft und Abbiegevorgänge verwendet.
Hierbei wird ein Abbiegevorgang durch den Winkel zwischen zwei Straßensegmenten an einer Kreuzung ermittelt.
Bisher erhält jeder Abbiegevorgang den gleichen Penalty, obwohl manche Kurven schneller befahren werden können.
Kurven innerhalb eines zusammenhängenden Segmentes werden bisher noch nicht betrachtet, sollten jedoch für realistischere Ergebnisse ebenfalls als Abbiegevorgang registriert werden.

Darüber hinaus muss der Öffnungswinkel einer Kurve mit in die Berechnung der zusätzlichen Zeit einfließen, da Abbiegungen mit größerem Winkel ebenfalls schneller befahren werden können.
Gleichzeitig muss auch die Geschwindigkeit des vorherigen und folgenden Straßensegments beachtet werden, da diese die Brems- bzw. Beschleunigungsstrecke alterieren.
Die Steigung eines Straßensegments hat einen signifikanten Einfluss auf die benötigte Zeit und muss daher ebenfalls einkalkuliert werden.
Es werden für die Implementierung dieser Funktionen noch weitere Analysen zu speziellen Testfahrten für das jeweilige Szenario (Steigung, Abbiegevorgang) benötigt.

Die Fahrzeugdimensionen und die Höchstgeschwindigkeit sind nicht im Backend fest codiert, sondern können als Parameter bei der Abfrage mitgeliefert werden.
Damit ist eine Erweiterung für andere Fahrzeugklassen sowohl der Feuerwehr, als auch des Rettungsdienstes oder der Polizei möglich, da für alle dieselbe Grundprämisse gilt (~\ref{cit:STVO}).

Im Hinblick auf \gls{osm}-Daten kann keine endgültige Sicherheit gewährleistet werden, da die Ergebnisse höchstens so gut wie die Daten sein können.

Zur Erstellung von Einzugsgebieten für das Profil für Einsatzfahrzeug, muss letztlich die Gewichtung für den Isochrones- Service verfügbar gemacht werden und wird als nächster Schritt implementiert.

\begin{figure}[htb]
\centering
\caption[Hydranten-Informationen in Heidelberg]
{\centering Hydranten-Informationen in Heidelberg
\par\justifying\scriptsize
Jeder rote Punkt in dieser Abbildung stellt den Eintrag einens Hydranten in der OSM-Datenbank dar. Durch das OpenFireMap Projekt wurden viele Städte flächendeckend mit Informationen, die für die Feuerwehr wichtig sind, versehen.
}
\label{fig:firehydrants}
\includegraphics[width = 0.80 \textwidth]{../media/firehydrants.png} \\
\end{figure}

Das Analysieren der Teststrecken ist durch das einzelne Senden der Anfragen und das anschließende Extrahieren der benötigten Werte zeitaufwändig.
Es wäre demnach hilfreich einen Service zum gleichzeitigen Senden von allen Test-Anfragen zu entwickeln, mit welchem danach ebenso die Antworten adäquat dargestellt werden können.
Dabei sollten zum Beispiel die Distanzen zwischen wichtigen Änderungspunkten auf einer Route berechnet werden können und tabellarisch aufgelistet werden.

Ein weiterer interessanter Aspekt den es zu beleuchten gilt, ist das unterschiedliche Verkehrsaufkommen auf einer Straße oder in einer Fußgängerzone.
So könnten Geschwindigkeiten für bestimmte Wegsegmente, je nach Zeit, an Berufsverkehr oder Tageszeit angepasst werden.
Nachts sollten beispielsweise generell weniger Autos auf der Straße sein, wodurch Zeit gespart wird, umgekehrt beim Berufsverkehr.

Die im Fragebogen erwähnte Suche nach Löschwasser-Quellen am Zielort kann und sollte verwirklicht werden.
In der \gls{osm}-Datengrundlage existieren bereits diverse Tags für Objekte, die von Einsatzkräften genutzt werden können.
Für die Feuerwehr besonders interessant ist zum Beispiel der \texttt{emergency=fire\_hydrant} Tag für Hydranten.
In vielen Städten sind diese durch das OpenFireMap-Projekt\footnote{https://wiki.openstreetmap.org/wiki/DE:OpenFireMap} flächendeckend digitalisiert und der \gls{osm}-Datenbank zur Verfügung gestellt worden (Abb.~\ref{fig:firehydrants}).