\section{Fazit}

Das erstellte Profil für Löschfahrzeuge wird als Prototyp betrachtet und kann bereits die Grundlegenden geforderten Eigenschaften erfüllen.
Es werden bei der Suche nach dem kürzesten Weg in einem Straßennetz die für andere Fahrzeuge geltenden Geschwindigkeitslimits auf geeignetere Werte angehoben.

Es ist für das Profil möglich Einbahnstraßen in Beide Richtungen zu durchfahren.
Bisher wird deswegen allerdings ab und zu die Gegenfahrbahn über längere Distanzen wegen einem Gewinn von nur wenigen Sekunden verwendet.
Darüber hinaus würde ein Penalty dem Risiko des Gegenverkehrs gerecht werden.
Interessant wäre es daher einen Penalty-Faktor einzubauen, der bei einer Nutzung entgegen der Fahrtrichtung eine Verzögerung mit einberechnet.
Dadurch würden auch in der Nähe liegende Parallelstraßen vorgezogen werden.
Darüber hinaus werden bisher nicht befahrbare Straßentypen wie Fußgängerzonen und Radwege für das Profil verfügbar gemacht.
Eine Routenführung bis zum Zielpunkt wird so gegenüber den normalen Fahrzeug Profilen in den meisten Fällen erreicht.

Derzeit wird die Schwerfälligkeit des Löschfahrzeuges durch positive Faktorisierung\footnote{Die Gewichtung wird mit Werten größer 1.0 multipliziert} der Gewichtung für Start, Ankunft und Turns bei der Suche nach dem schnellsten Weg berücksichtigt.
Für die durchgeführten Teststrecken wurde beim routen von Start bis Endpunkt jeweils genau die richtige Route zurückgegeben.
Ob diese Faktorisierung allgemein gültig ist, muss noch durch weitere Analysen überprüft werden.
Für die Berechnung der Zeit werden zur Berücksichtigung der Schwerfälligkeit einfache Penalties für Start, Ankunft und Turns verwendet.
Hierbei wird ein Turn durch den Winkel zwischen zwei Straßensegmenten an einer Kreuzung ermittelt.
Bisher erhält jeder Turn den gleichen Penalty obwohl manche Kurven schneller gefahren werden können.
Kurven innerhalb eines Segmentes werden bisher noch nicht registriert.

Für realistischere Ergebnisse müssen diese Kurven ebenfalls als Turn registriert werden.
Darüber hinaus muss der Öffnungswinkel einer Kurve mit in die Berechnung der zusätzlichen Zeit einfließen, da Turns mit größerem Winkel auch schneller gefahren werden können.
Gleichzeitig muss auch die Geschwindigkeit des vorherigen und folgenden Straßensegments beachtet werden, da diese die Brems- bzw. Beschleunigungsstrecke alterieren.
Die Steigung eines Straßensegments hat einen signifikannten Einfluss auf die benötigte Zeit und muss daher ebenfalls einkalkuliert werden.
Es werden für die Implementierung dieser Funktionen noch weitere Analysen zu speziellen Testfahren für das jeweilige Szenario (Steigung, Turn) benötigt.

Die Fahrzeugdimensionen und die Höchstgeschwindigkeit sind nicht im Backend festcodiert sondern können als Parameter bei der Abfrage mitgeliefert werden.
Damit ist eine Erweiterung für andere Fahrzeugklassen der Feuerwehr als auch des Rettungsdienstes oder der Polizei möglich, da für alle die selbe Grundprämisse gilt (~\ref{cit:STVO}).

Im Hinblick auf \gls{osm}-Daten kann keine endgültige Sicherheit gewährleistet werden, da die Ergebnisse höchstens so gut wie die Daten sein können

Für die Erstellung von Einzugsgebieten für das Emergency Profil, muss letztlich die Gewichtung für den Isochrones Service verfügbar gemacht werden.
Das ist in der bisherigen Implementierung noch nicht gelungen.

\newpage
\section{Ausblick}

\todo{Add Ausblick to Fazit?}

Das Analysieren der Teststrecken ist durch das einzelne Senden der Requests und das anschließende Extrahieren der benötigten Werte relativ zeitaufwändig.
Es wäre hilfreich ein Werkzeug zum gleichzeitigen senden von allen Testrequests zu entwickeln mit welchem danach ebenso die Antworten adäquat dargestellt werden können.
Dabei sollten zum Beispiel die Distanzen zwischen wichtigen Änderungspunkten auf einer Route berechnet werden können und direkt Tabellarisch aufgelistet werden.

Ein weiterer interessanter Aspekt den es zu beleuchten gilt, ist das unterschiedliche Verkehrsaufkommen auf einer Straße oder Fußgängerzone.
So könnten Geschwindigkeiten für bestimmte Wegsegmente je nach Zeit anhand von Berufsverkehr oder Tageszeit angepasst werden.
Nachts sollten zum Beispiel generell weniger Autos unterwegs sein, wodurch Zeit gespaart wird, Umgekehrt zur Rush-Hour.

Die im Fragebogen erwähnte Suche nach Löschwasser Quellen am Zielort kann und sollte verwirklicht werden.
In der \gls{osm} Datengrundlage existieren bereits diverse Tags für Objekte die von Einsatzkräften genutzt werden können.
Für die Feuerwehr besonders interessant ist zum Beispiel der \texttt{emergency=fire\_hydrant} Tag für Hydranten.
In vielen Städten sind diese durch das OpenFireMap-Projekt\footnote{https://wiki.openstreetmap.org/wiki/DE:OpenFireMap} flächendeckend abdigitalisiert und der OSM Datenbank zugefügt worden.