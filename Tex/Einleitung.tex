\section{Einleitung}

\subsection{Motivation}
Der \gls{ors} ist ein Projekt der Geoinformatik-Abteilung des geographischen Instituts der Ruprecht-Karls-Universität Heidelberg.
Dieser bietet, neben einer Fülle von Webdiensten, einen Isochronen-Service an, welcher zur Berechnung unterschiedliche Routing-Profile nutzen muss.\par
Mithilfe von Isochronen können Orte berechnet werden, die von einem Standort aus innerhalb einer bestimmten Zeit zu erreichen sind.
Für Unternehmen ist solch eine Analyse zum Beispiel bei der Standortwahl zur Berechnung von Umsatzpotentialen interessant.
Arbeitnehmer können mithilfe von Isochronen geeignete Wohnorte für eine zukünftige Arbeitsstelle ermitteln.
Auch der öffentliche Personennahverkehr richtet an Isochronen seine Verkehrsnetze aus oder legt Tarifzonen damit fest.
Für Polizei, Rettungsdienst sowie Feuerwehr geht es vor allem um das Einhalten amtlich vorgegebener zeitlicher Hilfsfristen.
Diese stellen einen bedeutenden Bestandteil für die Planung und Qualität der Einsätze von Feuerwehr und Rettungsdienst dar.\par
Da der Brandschutz im Gegensatz zum Rettungsdienst eine kommunale Aufgabe ist und nur in manchen Bundesländern bestimmten Standards unterliegt (\cite{bedarfsplan}), nutzt die Feuerwehr unterschiedliche Hilfsmittel, um Bedarfspläne für ihren Standort zu erstellen.
Der Isochronen-Service wird in diesem Zusammenhang bereits von unterschiedlichen Feuerwehreinheiten genutzt, allerdings bietet der Dienst in seiner bisherigen Form noch nicht alle erforderlichen Anforderungen für Einsatzfahrzeuge.
Um diese Nachteile zu überwinden, wurde eine Forschungskooperation mit der \gls{ffl} initiiert.
\vspace{1.2cm}

\textbf{§35 Abs. 1 StVO:}
\begin{quotation}
\label{cit:STVO}
{\itshape\rmfamily ''Von den Vorschriften dieser Verordnung sind die Bundeswehr, die Bundespolizei, die Feuerwehr, der Katastrophenschutz, die Polizei und der Zolldienst befreit, soweit das zur Erfüllung hoheitlicher Aufgaben dringend geboten ist.''}
\end{quotation}

\vspace{1.2cm}

Dieser kurze Absatz der Straßenverkehrs-Ordnung (StVO) ermöglicht es Einsatzfahrzeugen sich, unter Benutzung von Martinshorn und Blaulicht, über jede Vorschrift im Straßenverkehr hinwegzusetzen.
In einem Notfall hat das schnellste Erreichen des Zielorts eine höhere Priorität als Geschwindigkeitsbegrenzungen oder Fahrverbote.
Trotz einer großen Anzahl an Routing-Services auf dem Markt gibt es noch keinen, der diese Zusammenhänge berücksichtigt.

\subsection{Zielsetzung}
Das Ziel der Arbeit ist die Erstellung eines Routing Profils für Feuerwehrfahrzeuge auf der Basis von \gls{osm}-Daten unter der Berücksichtigung des \textit{§35 Abs.1 der StVO}. Dazu wurde eine Forschungskooperation mit der \gls{ffl} eingegangen. Dabei wird untersucht, bis zu welchem Grad diese Notstandsvollmachten im Ernstfall in Anspruch genommen werden können.
Auf Basis dieser, aus der Kooperation mit der \gls{ffl} ermittelten, Ergebnisse wird ein prototypisches Routing-Profil für Einsatzfahrzeuge entwickelt.
Die Implementierung wird im Rahmen der Bachelorarbeit auf einen Fahrzeugtyp der Feuerwehr begrenzt.
Die für diese Arbeit in Betracht gezogenen Löschfahrzeuge gehören den Klassen LF8, LF8/6 und MLF\footnote{\textit{LF}=Löschgruppenfahrzeug; \textit{MLF}=Mittleres Löschfahrzeug} an und wiegen je nach Beladung zwischen 3,5 und 7,5 Tonnen.
Diese Typen wurden von der \gls{ffl} als erste Priorität empfohlen.
Des Weitern wird bei der Erstellung des Routing-Profils darauf geachtet, dass eine Erweiterung für andere Einsatzfahrzeuge einfach möglich sind.
\vspace{0.5cm}

Als Basis wird das Profil auf dem Backend\footnote{Programm und Datenbanken die sich zur Berechnung von Routen auf einem über das Internet ansprechbaren Server befinden} des bereits bestehenden Routing-Service des \gls{ors} aufgebaut.
Zusätzlich sollen Java-Funktionen implementiert werden, die speziell auf das Profil für Einsatzfahrzeuge zugeschnitten sind.
Zur Darstellung wird das \gls{ors}-Frontend\footnote{Graphische Benutzeroberfläche der \gls{ors}-Website mit der Anfragen an das Backend gestellt und die Antworten dargestellt werden können} mit Hilfe der Javascript Programmiersprache angepasst.
Dadurch können die Ergebnisse in verständlicher und anschaulicher Weise präsentiert werden.