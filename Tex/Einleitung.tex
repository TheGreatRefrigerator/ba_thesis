\section{Einleitung}

\subsection{Motivation}
Zum Zeitpunkt der Erstellung dieser Arbeit ist der Verfasser in der Geoinformatik Abteilung des Geographischen Instituts der Ruprecht-Karls-Universität Heidelberg als wissenschaftliche Hilfskraft im Projekt \gls{ors} tätig.
Dieser bietet neben einer Fülle von Webdiensten, wie Geocoding und Routing, auch einen Isochronen-Service an.
Mit Isochronen werden Orte bestimmt, die von einem Standort aus in einer bestimmten Zeit erreicht werden können.
Für Unternehmen ist so eine Analyse zum Beispiel bei der Standortwahl zur Berechnung von Umsatzpotentialen interessant.
Arbeitnehmer können über Isochronen geeignete Wohnorte für eine zukünftige Arbeitsstelle ermitteln.
Bus und Bahn richten an Isochronen ihre Verkehrsnetze aus oder legen Tarifzonen damit fest.
Für Polizei, Rettungsdienst sowie Feuerwehr geht es vor allem um das Einhalten amtlich vorgegebener zeitlicher Hilfsfristen.
Diese stellen eine bedeutende Eigenschaft für die Planung und Qualität der Einsätze von Feuerwehr und Rettungsdienst dar.\par
Der Brandschutz ist im Gegensatz zum Rettungsdienst eine kommunale Aufgabe und unterliegt nur in manchen Bundesländern bestimmten Standards (\cite{bedarfsplan}).
Daher bedienen sich diese Organisationen unterschiedlicher Hilfsmittel um Bedarfspläne für ihren Standort zu erstellen.
\medskip

Da mit dem Isochronen-Service jene zeitlichen Erreichbarkeitsanalysen durchgeführt werden können, ist dieser für die Erstellung eines Brandschutzbedarfsplans der Feuerwehr geeignet.
Jedoch kann der Dienst in seiner bisherigen Form noch nicht alle erforderlichen Anforderungen für Einsatzfahrzeuge erfüllen.
Um diese Nachteile zu überwinden, wurde eine Forschungskooperation mit der \gls{ffl} eingegangen.
\vspace{1.2cm}

\textbf{§35 Abs. 1 StVO:}
\begin{quotation}
\label{cit:STVO}
{\itshape\rmfamily ''Von den Vorschriften dieser Verordnung sind die Bundeswehr, die Bundespolizei, die Feuerwehr, der Katastrophenschutz, die Polizei und der Zolldienst befreit, soweit das zur Erfüllung hoheitlicher Aufgaben dringend geboten ist.''}
\end{quotation}

\vspace{1.2cm}

Dieser kurze Absatz der Straßenverkehrs-Ordnung ermöglicht Einsatzfahrzeugen sich unter Benutzung von Martinshorn und Blaulicht über jede Vorschrift im Straßenverkehr hinwegzusetzen.
In einem Notfall hat das schnellste Erreichen des Zielorts eine höhere Priorität als Geschwindigkeitsbegrenzungen oder Fahrverbote.
Bisher gibt es trotz einer großen Anzahl an Routing-Services keinen, der diese Tatsache berücksichtigt.

\subsection{Zielsetzung}
Das Ziel dieser Arbeit ist in Kooperation mit der \gls{ffl} zu Ermitteln, bis zu welchem Grad diese Notstandsvollmachten im Ernstfall in Anspruch genommen werden können.
Auf Basis dieser Informationen soll dann ein prototypisches Routing-Profil für Einsatzfahrzeuge entwickelt werden.
Die Implementierung wird aufgrund des Umfangs einer Bachelorarbeit auf einen Fahrzeugtyp der Feuerwehr begrenzt.
Dabei handelt es sich um Löschfahrzeuge der Klassen LF8, LF8/6 und MLF zwischen 3,5 und 7,5 Tonnen (je nach Beladung).
Diese Typen wurden von der \gls{ffl} als erste Priorität empfohlen.
Allerdings wird bei der Erstellung dieses Routing-Profils darauf geachtet, dass Erweiterungen für diverse Einsatzfahrzeuge einfach möglich sind.
\vspace{0.5cm}

Als Basis wird das Profil auf dem Backend\footnote{Programm und Datenbanken die sich zur Berechnung von Routen auf einem über das Internet ansprechbaren Server befinden} des bereits bestehenden Routing Service des \gls{ors} aufgebaut.
Zusätzlich sollen Java Funktionen implementiert werden, die speziell auf das Emergency Profil zugeschnitten sind.
Zur Darstellung wird das \gls{ors} Frontend\footnote{Graphische Benutzeroberfläche der \gls{ors}-Website mit der Anfragen an das Backend gestellt und die Antworten dargestellt werden können} mit Hilfe der Java-Script Programmiersprache angepasst.
Dadurch können die Ergebnisse in verständlicher und anschaulicher Weise präsentiert werden.