{\centering\section*{Abstract}}

\vspace{1cm}

This study aims to develop a prototype routing profile for fire engines.
\medskip

The research is carried out with the voluntary fire department of Lützelburg.
The existing routing services are not efficient for emergency routings since for example they do not consider that emergency vehicles are allowed to ignore the StVO\footnote{german road traffic regulation}.
This study presents the development of an emergency routing profile as an extention of OpenRouteService, a routing engine based on OpenStreetMap database.
The new emergency routing profile considers reasonable speed of emergency vehicles and further allows the possibility of navigating through cycleways, pedestrian zones and other possible routes that should not be used by vehicles in normal situations.
Other special characteristics such as the possibility of driving in the opposite direction of a one-way road makes the routing profile efficient and applicable for emergency routing.
As another example, the time delay as cause of acceleration and break processes is taken into account with calculating time penalties for starting, turnings and arrival.
Simulations were done in order to evaluate and calibrate the routing profile through analytical comparison.\par
The results show that \newline $a)$ the functionality to consider road incline is necessary and should be included in future versions of the routing profile and \newline $b)$ the differences in travel time between the test runs and the profile are less than 10 seconds for most of the waypoints on the test route.
\bigskip

The present study provides the basis for an efficient routing and travel time calculation of emergency vehicles as well as assistance in the development of demand plans.