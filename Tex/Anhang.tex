\section{Anhang}
\label{sec:anhang}
\subsection*{Copyrights}
Die Rechte für die mit geojson.io generierten Abbildungen(\ref{fig:footway}, \ref{fig:school}, \ref{fig:ramp}, \ref{fig:oneway}, \ref{fig:opposite}) gehen an \copyright Mapbox für den Kartenstil und \copyright \gls{osm} für die Daten.
\newline Die Rechte der mit dem \gls{ors} entworfenen Abbildungen(\ref{fig:isochrones}, \ref{fig:drive1}, \ref{fig:drive2}, \ref{fig:drive3}, \ref{fig:steig}, \ref{fig:curve}, \ref{fig:temp30}, \ref{fig:turnosm}) gehen an \copyright \gls{osm} contributors für die Daten und \copyright Maxim Rylov für den Kartenstil.
Für Abbildung \ref{fig:turnworld} liegt das Copyright bei \copyright 2017 Google für die Bilder und \copyright 2017 GeoBasis-DE/BKG für die Kartendaten.
Für Abbildung \ref{fig:traintunnel} liegt das Copyright bei \copyright DigitalGlobe.

\subsection*{Links}
\textbf{Fragenkatalog:}
\sloppy
\url{https://docs.google.com/document/d/1nwjmea0jwauJWezk\_2TMQs5CbAHv7he2NI4kxqMu2w4/}
\medskip

\textbf{Emergency Routing Service:}

\url{http://emergency.openrouteservice.org/directions?n1=48.448646\&n2=10.826855\&n3=14\&b=0\&c=0\&k1=en-US\&k2=km}
\medskip

\textbf{Testfahrt 1:}

GeoJson-Antwort:\par
\url{https://disaster-api.openrouteservice.org/emergency/routes?api_key=58d904a497c67e00015b45fcbd837ca3e137425f653e26a676ecd396&attributes=detourfactor|percentage&coordinates=10.824569,48.454111|10.789776,48.466555&elevation=true&extra_info=steepness|waytype|surface|avgspeed&geometry=true&geometry_format=geojson&instructions=true&instructions_format=html&language=en-US&options={"profile_params":{"restrictions":{"width":"2.5","height":"3","weight":"7.5","length":"7"}},"maximum_speed":"80"}\&preference=fastest\&profile=driving-emergency\&units=m}
\medskip

Anzeige im \gls{ors} Client:\par
\url{http://emergency.openrouteservice.org/directions?n1=48.454186&n2=10.825616&n3=18&a=48.454111,10.824569,48.466555,10.789776&b=5b&c=0&f3=3&f1=7.5&f2=2.5&f5=7&d=80&k1=en-US&k2=km}
\medskip

\textbf{Testfahrt 2:}

GeoJson-Antwort:\par
\url{https://disaster-api.openrouteservice.org/emergency/routes?api_key=58d904a497c67e00015b45fcbd837ca3e137425f653e26a676ecd396&attributes=detourfactor|percentage&coordinates=10.824569,48.454111|10.85755,48.457434&elevation=true&extra_info=steepness|waytype|surface|avgspeed&geometry=true&geometry_format=geojson&instructions=true&instructions_format=html&language=en-US&options={"profile_params":{"restrictions":{"width":"2.5","height":"3","weight":"7.5","length":"7"}},"maximum_speed":"80"}&preference=fastest\&profile=driving-emergency&units=m}
\medskip

Anzeige im \gls{ors} Client:\par
\url{http://emergency.openrouteservice.org/directions?n1=48.454236&n2=10.826409&n3=18&a=48.454111,10.824569,48.457434,10.85755&b=5b&c=0&d=80&f3=3&f1=7.5&f2=2.5&f5=7&k1=en-US&k2=km}
\medskip

\textbf{Testfahrt 3:}

GeoJson-Antwort:\par
\url{https://disaster-api.openrouteservice.org/emergency/routes?api_key=58d904a497c67e00015b45fcbd837ca3e137425f653e26a676ecd396&attributes=detourfactor|percentage&coordinates=10.824569,48.454111|10.805097,48.443849&elevation=true&extra_info=steepness|waytype|surface|avgspeed&geometry=true&geometry_format=geojson&instructions=true&instructions_format=html&language=en-US&options={"profile_params":{"restrictions":{"width":"2.5","height":"3","weight":"7.5","length":"7"}},"maximum_speed":"80"}&preference=fastest&profile=driving-emergency&units=m}
\medskip

Anzeige im \gls{ors} Client:\par
\url{http://emergency.openrouteservice.org/directions?n1=48.44654&n2=10.826383&n3=15&a=48.454111,10.824569,48.443849,10.805097&b=5b&c=0&d=80&f3=3&f1=7.5&f2=2.5&f5=7&k1=en-US&k2=km}
\medskip

\subsection*{Sourcecode}
\label{sec:source}

\sloppy
\textbf{Komplettes Backend:}\newline
\url{https://github.com/TheGreatRefrigerator/openrouteservice/tree/emergencyrouting}
\newline
\smallskip
\textbf{Erstellte Java-Klassen}\newline
\url{https://github.com/TheGreatRefrigerator/ba_thesis/tree/master/data/JavaFiles}
\newline
\medskip
\textbf{Frontend:}\newline
\url{https://github.com/GIScience/openrouteservice-app/tree/emergencyrouting}
\fussy

\newpage
\section*{Erklärung}
\vspace{1cm}
Hiermit versichere ich, dass ich die vorliegende Bachelorarbeit selbstständig verfasst, noch nicht anderweitig zu Prüfungszwecken vorgelegt, keine anderen als die angegebenen Quellen oder Hilfsmittel benutzt, sowie wörtliche und sinngemäße Zitate als solche gekennzeichnet habe.\par
\bigskip

{\flushleft Heidelberg den \today }
%{\hfill\vspace*{1cm}\includegraphics[width = 0.30\textwidth]{../media/unterschrift.jpg}}
{\flushright\vspace{-0.4cm}\hfill (Amandus Butzer)      }

\newpage
\section*{Danksagungen}

An dieser Stelle möchte ich allen Personen danken, die dazu beigetragen haben diese Arbeit zu verwirklichen.\par
\vspace{0.5cm}
Ein großes Dankeschön geht natürlich an die Freiwillige Feuerwehr Lützelburg für die Bereitstellung des Fahrzeuges.
Insbesondere möchte ich Stefan Witossek für die detaillierten Informationen sowie die Organisation und Durchführung aller Testfahrten danken.
Ohne ihn wäre diese Arbeit nicht möglich gewesen.\par
\vspace{0.5cm}
Danke an meine Kollegen vom Openrouteservice Team Timothy: Ellersiek, Maximilian Rylov und Lu Liu, die mir mit motivierenden Gesprächen und hilfreichen Erklärungen zur Seite gestanden sind und mich in jeder Hinsicht unterstützt haben.\par
\vspace{0.5cm}
Bei meinem Betreuer Prof. Dr. Alexander Zipf möchte ich mich für das Prüfen der Arbeit und der Möglichkeit bedanken beim Openrouteservice zu arbeiten.\par
Ebenfalls bedanken möchte ich mich bei Prof. Dr. Jo\~ao Porto de Albuquerque, der trotz eines mit internationalen Reisen gefüllten Terminkalenders Zeit für das Prüfen meiner Arbeit gefunden hat.\par
\vspace{0.5cm}
Außerdem möchte ich mich bei meinen Kommilitonen Angelika Kinas und Marcel Reinmuth sowie meinen Freunden Natalie Scherer und Amin Mobasheri bedanken, die mich in dieser Zeit mit unterhaltsamen Mittagspausen, motivierenden Gesprächen, Tipps und dem Korrekturlesen der Arbeit besonders unterstützt haben.
\vspace{0.5cm}
Ein letztes Dankeschön geht an meinen Bruder Ruben Butzer und meinen Vater Friedhelm Butzer für Tipps zum setzen von Kommata und das finale Korrekturlesen.

\newpage
\vspace{1cm}
%last page