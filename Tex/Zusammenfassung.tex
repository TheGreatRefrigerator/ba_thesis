{\centering\section*{Zusammenfassung}}

\vspace{1cm}

Ziel der vorliegenden Bachelorarbeit war es ein prototypisches Routing-Profil für Löschfahrzeuge, in einer Forschungskooperation mit der Freiwilligen Feuerwehr Lützelburg, zu entwickeln.
\medskip

Gegenwärtige Routing-Services sind für ein Routing im Notfall nicht effizient, da sie nicht berücksichtigen, dass Einsatzfahrzeuge im Notfall alle Vorschriften der StVO vernachlässigen dürfen.
Auf Basis des Openrouteservice wurde mit OpenStreetMap als Datengrundlage ein Routing-Profil für Einsatzfahrzeuge entwickelt, welches für Einsatzfahrzeuge adäquate Geschwindigkeiten verwendet sowie Radwege, Fußgängerzonen und weitere Wegtypen befahren darf. Weiterhin dürfen Einbahnstraßen in Gegenrichtung durchfahren werden.
Die Verzögerung durch Beschleunigungs- und Abbremsvorgänge wurde anhand von Zeit-Strafen für Anfahrt, Abbiegevorgänge und Ankunft berücksichtigt.
Es wurden Testfahrten durchgeführt um das Profil zu kalibrieren und mittels analytischer Vergleiche zwischen den Fahrtzeiten zu validieren.\par
Die Ergebnisse zeigen, dass \newline $a)$ eine Funktion zur Berücksichtigung der Steigung in einer zukünftigen Version des Routing-Profils eingebaut werden muss und \newline $b)$ die Fahrtzeitunterschiede zwischen den Testfahrten und dem Profil für die meisten Wegpunkte der Teststrecken im 10-Sekunden-Bereich liegen.
\bigskip

Die vorliegende Arbeit bildet eine Grundlage für effiziente Routenfindung und Fahrzeitberechnung für Notfallfahrzeuge und bietet darüber hinaus Hilfestellung bei der Entwicklung von Bedarfsplänen.